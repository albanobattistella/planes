\documentclass{report}
\usepackage{listings}
\usepackage{xcolor}
\lstset { %
    language=C++,
    backgroundcolor=\color{black!5}, % set backgroundcolor
    basicstyle=\footnotesize,% basic font setting
    breaklines
}
\usepackage{graphicx}
\usepackage{url}
\usepackage{hyperref}
\hypersetup{
    colorlinks=true,
    linkcolor=blue,
    filecolor=magenta,      
    urlcolor=cyan,
}


\title{Development of the Planes Game, a Version of the Battleships Game}
\date{2017-06-09}
\author{Cristian Cucu}
\begin{document}
\maketitle
\newpage
\lstset{language=C++}
\tableofcontents{}

\chapter{Introduction}

\section{Game Description}
The game is a variant of the classical \href{https://en.wikipedia.org/wiki/Battleship_(game)}{battleship game}. The ships will be here called planes and are shown in \ref{fig:board}.
\begin{figure}[h]
  \includegraphics[width = \textwidth]{BoardWithPlanes.png}
  \caption{Board game with 3 planes}
  \label{fig:board}
\end{figure}
The project is organized in two main parts: the game engine and the graphical user interfaces. The game engine is meant to be implemented as a library such it can be used with a number of graphical frontents. One of the design goals is to make this engine independent of a specific C++ library. Three Graphical User Interfaces (GUI) based on the Qt framework were programmed up to this point. They are called PlanesWidget, a simple QWidget approach which also includes debugging tools for the computer's strategy, PlanesGraphicsScene, a GUI based on the QGraphicsScene API, and PlanesQML, based on the QML engine. Another graphical user interface was programmed with Java and interfaces through the Java Native Interface to the C++ game engine.

The complete source code of the projects can be found in GitHub (\url{https://github.com/xxxcucus/planes}).

\chapter {The Game Engine }
\section{Requirements Analysis}
We need an object that describes a plane which should at least contain information about the position of the plane on the game grid, the orientation of the plane, the shape of a plane. Additionaly we would need a game board/grid object. It should not be restricted to a specific geometry, should know where each plane is positioned and how many planes there are. Since the game is played against the computer there should be a kind of strategy object that decides the computer's next move. The organization of the game into a series of human vs. computer rounds needs also to be modelled in code.

\include{ThePlaneObject}
\include{ThePlanePointIterator}
\include{TheComputerStrategy}
\include{TheGameBoard}
\include{TheGameController}

\chapter {The Graphical User Interface}

\section{Introduction}

We have currently created 5 Graphical User Interfaces (GUIs) three based on C++ and the Qt Framework and two based on Java: one with the JavaFx library and one with a Java Android app. From the 3 C++ GUIs PlanesWidget uses simple Widget technology of Qt, PlanesGraphicsScene uses the QGraphicsScene/QGraphicsView functionality of Qt, and PlanesQML uses the QML engine of Qt. The chronological order of developing the GUIs was: PlanesWidget, PlanesGraphicsScene, PlanesQML, PlanesJavaFx, PlanesAndroid. As we moved from GUI to GUI we made the separation between the GUI and the GameEngine deeper and deeper. We will comment on these aspects as we present each of the GUI. All of these graphical interfaces are targeted to desktop applications.

Each GUI consist mainly of two parts: a left pane (in most projects denoted as LeftPane) containing the controls for positioning planes in the board editing phase of the game, the moves statistics during a round, the global score and a button allowing to start a new round at the end of a round, and a right pane (in most projects denoted as RightPane) which contains the player's and computer's game boards as well as a help file. In PlanesWidget this structure is not completely respected and we will comment on that as we will present each of the GUI concepts.

An essential aspect of the graphical user interface is how the board games are displayed. 

\section {PlanesAndroid}
\include{PlanesJavaFx}
\include{PlanesQML}
\include{PlanesGraphicsScene}
\include{PlanesWidget}

\end{document}